%  LaTeX support: latex@mdpi.com
%  In case you need support, please attach all files that are necessary for compiling as well as the log file, and specify the details of your LaTeX setup (which operating system and LaTeX version / tools you are using).

%=================================================================
\documentclass[water,article,submit,moreauthors,pdftex]{mdpi}

% If you would like to post an early version of this manuscript as a preprint, you may use preprint as the journal and change 'submit' to 'accept'. The document class line would be, e.g., \documentclass[preprints,article,accept,moreauthors,pdftex]{mdpi}. This is especially recommended for submission to arXiv, where line numbers should be removed before posting. For preprints.org, the editorial staff will make this change immediately prior to posting.

%% Some pieces required from the pandoc template
\providecommand{\tightlist}{%
  \setlength{\itemsep}{0pt}\setlength{\parskip}{4pt}}
\setlist[itemize]{leftmargin=*,labelsep=5.8mm}
\setlist[enumerate]{leftmargin=*,labelsep=4.9mm}

\usepackage{longtable}

% see https://stackoverflow.com/a/47122900

%--------------------
% Class Options:
%--------------------
%----------
% journal
%----------
% Choose between the following MDPI journals:
% acoustics, actuators, addictions, admsci, aerospace, agriculture, agriengineering, agronomy, algorithms, animals, antibiotics, antibodies, antioxidants, applsci, arts, asc, asi, atmosphere, atoms, axioms, batteries, bdcc, behavsci , beverages, bioengineering, biology, biomedicines, biomimetics, biomolecules, biosensors, brainsci , buildings, cancers, carbon , catalysts, cells, ceramics, challenges, chemengineering, chemistry, chemosensors, children, cleantechnol, climate, clockssleep, cmd, coatings, colloids, computation, computers, condensedmatter, cosmetics, cryptography, crystals, dairy, data, dentistry, designs , diagnostics, diseases, diversity, drones, econometrics, economies, education, electrochem, electronics, energies, entropy, environments, epigenomes, est, fermentation, fibers, fire, fishes, fluids, foods, forecasting, forests, fractalfract, futureinternet, futurephys, galaxies, games, gastrointestdisord, gels, genealogy, genes, geohazards, geosciences, geriatrics, hazardousmatters, healthcare, heritage, highthroughput, horticulturae, humanities, hydrology, ijerph, ijfs, ijgi, ijms, ijns, ijtpp, informatics, information, infrastructures, inorganics, insects, instruments, inventions, iot, j, jcdd, jcm, jcp, jcs, jdb, jfb, jfmk, jimaging, jintelligence, jlpea, jmmp, jmse, jnt, jof, joitmc, jpm, jrfm, jsan, land, languages, laws, life, literature, logistics, lubricants, machines, magnetochemistry, make, marinedrugs, materials, mathematics, mca, medicina, medicines, medsci, membranes, metabolites, metals, microarrays, micromachines, microorganisms, minerals, modelling, molbank, molecules, mps, mti, nanomaterials, ncrna, neuroglia, nitrogen, notspecified, nutrients, ohbm, particles, pathogens, pharmaceuticals, pharmaceutics, pharmacy, philosophies, photonics, physics, plants, plasma, polymers, polysaccharides, preprints , proceedings, processes, proteomes, psych, publications, quantumrep, quaternary, qubs, reactions, recycling, religions, remotesensing, reports, resources, risks, robotics, safety, sci, scipharm, sensors, separations, sexes, signals, sinusitis, smartcities, sna, societies, socsci, soilsystems, sports, standards, stats, surfaces, surgeries, sustainability, symmetry, systems, technologies, test, toxics, toxins, tropicalmed, universe, urbansci, vaccines, vehicles, vetsci, vibration, viruses, vision, water, wem, wevj

%---------
% article
%---------
% The default type of manuscript is "article", but can be replaced by:
% abstract, addendum, article, benchmark, book, bookreview, briefreport, casereport, changes, comment, commentary, communication, conceptpaper, conferenceproceedings, correction, conferencereport, expressionofconcern, extendedabstract, meetingreport, creative, datadescriptor, discussion, editorial, essay, erratum, hypothesis, interestingimages, letter, meetingreport, newbookreceived, obituary, opinion, projectreport, reply, retraction, review, perspective, protocol, shortnote, supfile, technicalnote, viewpoint
% supfile = supplementary materials

%----------
% submit
%----------
% The class option "submit" will be changed to "accept" by the Editorial Office when the paper is accepted. This will only make changes to the frontpage (e.g., the logo of the journal will get visible), the headings, and the copyright information. Also, line numbering will be removed. Journal info and pagination for accepted papers will also be assigned by the Editorial Office.

%------------------
% moreauthors
%------------------
% If there is only one author the class option oneauthor should be used. Otherwise use the class option moreauthors.

%---------
% pdftex
%---------
% The option pdftex is for use with pdfLaTeX. If eps figures are used, remove the option pdftex and use LaTeX and dvi2pdf.

%=================================================================
\firstpage{1}
\makeatletter
\setcounter{page}{\@firstpage}
\makeatother
\pubvolume{xx}
\issuenum{1}
\articlenumber{5}
\pubyear{2019}
\copyrightyear{2019}
%\externaleditor{Academic Editor: name}
\history{Received: date; Accepted: date; Published: date}
\updates{yes} % If there is an update available, un-comment this line

%% MDPI internal command: uncomment if new journal that already uses continuous page numbers
%\continuouspages{yes}

%------------------------------------------------------------------
% The following line should be uncommented if the LaTeX file is uploaded to arXiv.org
%\pdfoutput=1

%=================================================================
% Add packages and commands here. The following packages are loaded in our class file: fontenc, calc, indentfirst, fancyhdr, graphicx, lastpage, ifthen, lineno, float, amsmath, setspace, enumitem, mathpazo, booktabs, titlesec, etoolbox, amsthm, hyphenat, natbib, hyperref, footmisc, geometry, caption, url, mdframed, tabto, soul, multirow, microtype, tikz

%=================================================================
%% Please use the following mathematics environments: Theorem, Lemma, Corollary, Proposition, Characterization, Property, Problem, Example, ExamplesandDefinitions, Hypothesis, Remark, Definition
%% For proofs, please use the proof environment (the amsthm package is loaded by the MDPI class).

%=================================================================
% Full title of the paper (Capitalized)
\Title{Ethics Paper}

% Authors, for the paper (add full first names)
\Author{Audrey Bertin$^{1}$}

% Authors, for metadata in PDF
\AuthorNames{Audrey Bertin}

% Affiliations / Addresses (Add [1] after \address if there is only one affiliation.)
\address{%
$^{1}$ \quad Smith College - Department of Statistical and Data Sciences
Northampton, MA, USA; \\
}
% Contact information of the corresponding author
\corres{Correspondence: }

% Current address and/or shared authorship








% The commands \thirdnote{} till \eighthnote{} are available for further notes

% Simple summary
\simplesumm{A Simple summary goes here.}

% Abstract (Do not insert blank lines, i.e. \\)
\abstract{{[}Coming soon{]}}

% Keywords
\keyword{keyword 1; keyword 2; keyword 3 (list three to ten pertinent
keywords specific to the article, yet reasonably common within the
subject discipline.).}

% The fields PACS, MSC, and JEL may be left empty or commented out if not applicable
%\PACS{J0101}
%\MSC{}
%\JEL{}

%%%%%%%%%%%%%%%%%%%%%%%%%%%%%%%%%%%%%%%%%%
% Only for the journal Diversity
%\LSID{\url{http://}}

%%%%%%%%%%%%%%%%%%%%%%%%%%%%%%%%%%%%%%%%%%
% Only for the journal Applied Sciences:
%\featuredapplication{Authors are encouraged to provide a concise description of the specific application or a potential application of the work. This section is not mandatory.}
%%%%%%%%%%%%%%%%%%%%%%%%%%%%%%%%%%%%%%%%%%

%%%%%%%%%%%%%%%%%%%%%%%%%%%%%%%%%%%%%%%%%%
% Only for the journal Data:
%\dataset{DOI number or link to the deposited data set in cases where the data set is published or set to be published separately. If the data set is submitted and will be published as a supplement to this paper in the journal Data, this field will be filled by the editors of the journal. In this case, please make sure to submit the data set as a supplement when entering your manuscript into our manuscript editorial system.}

%\datasetlicense{license under which the data set is made available (CC0, CC-BY, CC-BY-SA, CC-BY-NC, etc.)}

%%%%%%%%%%%%%%%%%%%%%%%%%%%%%%%%%%%%%%%%%%
% Only for the journal Toxins
%\keycontribution{The breakthroughs or highlights of the manuscript. Authors can write one or two sentences to describe the most important part of the paper.}

%\setcounter{secnumdepth}{4}
%%%%%%%%%%%%%%%%%%%%%%%%%%%%%%%%%%%%%%%%%%

% Pandoc citation processing

\usepackage{float} \floatplacement{figure}{H}

\begin{document}
%%%%%%%%%%%%%%%%%%%%%%%%%%%%%%%%%%%%%%%%%%

The United States is currently facing a crisis of personal data privacy.
In recent decades, a new era in the digital world---commonly termed the
\emph{Internet of Things}, or IoT---has emerged. The IoT refers to the
devices and people who enable the sharing of data worldwide, and is used
to characterize the modern internet age as one whose focus is now on big
data.

Improvements in computing power and internet speed, alongside the
development of new technologies capable of storing and utilizing massive
quantities of data, have ushered in a new economic age: the data
economy. Data is now a hot commodity, with the power to be incredibly
valuable to those with the technology to utilize them. The Big Data
Strategist at Oracle---a major software company---once said that ``data
is in fact a new kind of capital on par with financial capital for
creating new products and services.'' \emph{(When data is capital:
Datafication, accumulation, and extraction)}. Data provides this value
through several means: by enabling businesses to profile and target
people, leading to higher success rates in attracting customers; by
providing information that can be used to help optimize systems; by
helping manage and control things; by allowing companies to model
probabilities more accurately; and by allowing certain software to
operate in a way that would not be possible otherwise. \emph{(When data
is capital: Datafication, accumulation, and extraction)}

Much of this data comes directly from the general public---the people
who use the goods and services produced by companies who participate in
the data economy, such as Amazon, Google, and Facebook. Consumers' data
is collected constantly. Amazon tracks users' purchases and voice
commands---even going so far as to track the lines highlighted in books
bought by Kindle readers.
(\url{https://www.theguardian.com/technology/2020/feb/03/amazon-kindle-data-reading-tracking-privacy}).
Google tracks every search users make, every YouTube video they watch,
their full calendar schedule, their Gmail messages,everywhere they go,
how long they stay there, and what route they take---even if Google Maps
is not open
(\url{https://medium.com/swlh/an-in-depth-look-into-all-the-ways-google-tracks-you-in-2019-b158acf05b29}).

This large-scale data collection poses significant ethical implications
when considering the potential effects on consumers. For one, there is
the concern of data breeches. Since user data is collected and shared
over the internet, it is at risk of being released---or taken for
nefarious purposes---through cyber-hacking events. Just last year,
hundreds of millions of facebook users' phone numbers, locations, and
emails were stolen
(\url{https://www.npr.org/2021/04/09/986005820/after-data-breach-exposes-530-million-facebook-says-it-will-not-notify-users}).
Facebook is not alone; many other companies have seen serious data
breaches in recent years: Yahoo, Experian, Twitter, and Microsoft, to
name a few
(\url{https://www.informationisbeautiful.net/visualizations/worlds-biggest-data-breaches-hacks/}).
This danger is heightened further in situations involving private health
or financial data, leaving consumers at risk of negative impacts from
the release of sensitive health information, as well as possible
identity theft and financial harm. Additionally, the misuse or unwanted
release of information on polarizing issues such as religion, sexual and
gender identity, or data indicating the use of abortion services could
potentially put vulnerable consumers in harms' way---making them targets
for attack.

There is also the danger that data, even in its intended use, can result
in harms to consumers. Large scale data collection of consumers is often
used to power machine learning algorithms, which use data to make
predictions about people. These algorithms---though seemingly objective
at first glance---are often negatively biased toward minoritized people
(use definition from Data Feminism \& cite!). For example, facial
recognition algorithms built in large part off of Facebook photos
misidentify black women at a significantly higher rate than other groups
(source).

Data can also be used for psychological manipulation. Sites like
Facebook and Twitter have a wealth of data on their users---enough to
predict with high accuracy how they will react when exposed to certain
stimuli. These sites can use that knowledge to spread targeted messages
and actively change the beliefs held the public. This is what happened
in the 2016 election, when Facebook's advertising system targeted those
individuals it calculated to be likely susceptible to conservative
messaging with advertisements that reflected the ideals of Donald Trump.
Though not known for certain, it is widely believed that these
advertisements may have convinced enough voters to support President
Trump that he eventually won the election. (Source)

Problem setup:

\begin{itemize}
\tightlist
\item
  In US, big challenge w/ information privacy:
\end{itemize}

\begin{enumerate}
\def\labelenumi{\arabic{enumi}.}
\item
  the internet of things
  (\url{https://lawdigitalcommons.bc.edu/cgi/viewcontent.cgi?referer=\&httpsredir=1\&article=3622\&context=bclr})
\item
  data commodification -- data as capital
  (\url{https://www.google.com/url?sa=t\&rct=j\&q=\&esrc=s\&source=web\&cd=\&ved=2ahUKEwiD8NynpIXwAhWnAZ0JHS2aDpsQFjACegQIAhAD\&url=https\%3A\%2F\%2Fresearch.monash.edu\%2Ffiles\%2F303893944\%2F303893762_oa.pdf\&usg=AOvVaw09nsGbgbaIQ7dJa-sP-ITn})
  (\url{https://columbialawreview.org/content/paying-for-privacy-and-the-personal-data-economy/})
\item
  Dangers:
\end{enumerate}

\begin{itemize}
\tightlist
\item
  data breaches + data used against you (financial info, identity theft,
  health data particularly damaging)
\item
  interest targeting a very powerful tool to cause dangerous
  psychological shifts (see FB scandal)
\item
  tools to enable exclusion (see sexual orientation scanner, ER triage
  algorithms)
\item
  bias
\end{itemize}

\url{https://lawdigitalcommons.bc.edu/cgi/viewcontent.cgi?referer=\&httpsredir=1\&article=3622\&context=bclr}

\begin{enumerate}
\def\labelenumi{\arabic{enumi}.}
\setcounter{enumi}{3}
\item
  Current protections in us:
  \url{https://columbialawreview.org/content/paying-for-privacy-and-the-personal-data-economy/}
  \url{https://iclg.com/practice-areas/data-protection-laws-and-regulations/usa}
  \url{https://www.cfr.org/report/reforming-us-approach-data-protection}
\item
  What other countries have done that we haven't
\end{enumerate}

\begin{itemize}
\tightlist
\item
  GDPR in EU \url{https://gdpr-info.eu}
\item
  South Korea
  \url{https://iapp.org/news/a/gdpr-matchup-south-koreas-personal-information-protection-act/}
  , \url{https://papers.ssrn.com/sol3/papers.cfm?abstract_id=2904896}
\item
  Chile
  \url{https://cms.law/en/int/expert-guides/cms-expert-guide-to-data-protection-and-cyber-security-laws/chile}
  \url{https://www.eff.org/deeplinks/2020/09/look-back-and-ahead-data-protection-latin-america-and-spain}
\end{itemize}

\begin{enumerate}
\def\labelenumi{\arabic{enumi}.}
\setcounter{enumi}{4}
\tightlist
\item
  Why it's hard for us to do the same
\end{enumerate}

\begin{itemize}
\tightlist
\item
  We don't currently have an agency that could be focused on managing
  data protection --- closest is FTC but they don't have oversight power
  of lots of private companies
\item
  Powerful tech lobby at federal level has huge leverage on voting
\item
  All the states are so different from one another so state-only laws
  would be a challenging patchwork if not controlled
\item
  potentially not enough public support. In US, we're just used to our
  data being used all the time. However, concern does seem to be
  growing.
\end{itemize}

\url{https://www.washingtonpost.com/news/powerpost/paloma/the-cybersecurity-202/2018/05/25/the-cybersecurity-202-why-a-privacy-law-like-gdpr-would-be-a-tough-sell-in-the-u-s/5b07038b1b326b492dd07e83/}

It COULD work though --- look at HIPAA!! It is super strict in the
healthcare industry, and goes pretty smoothly. tech companies have
adapted to make HIPAA compliant software

\begin{enumerate}
\def\labelenumi{\arabic{enumi}.}
\setcounter{enumi}{6}
\tightlist
\item
  What we should do
\end{enumerate}

DON'T TRY TO CREATE WHOLE PRIVACY LAW --- will not work in the current
climate. No way to enforce and considerably too big of a jump. Work in
smaller steps w/ less punishment

\begin{itemize}
\tightlist
\item
  Create a data protection agency
\item
  focus on prevention, not monetary punishment (better incentives for
  companies)
\item
  widen the definition of what can count as a data-privacy related harm
  to afford more opportunities to individuals to take problems to court
  as well as what is considered sensitive data
  \url{https://www.cfr.org/report/reforming-us-approach-data-protection}
\item
  On the statewide level, if wanting to create laws, do so with an eye
  for making them consistent. Once a majority of states have them, more
  pressure for federal change
\item
  Eventually do need federal law. Should attempt to get both Republican
  and Democratic support which apparently already exists
  (\url{https://www.theregreview.org/2021/03/13/saturday-seminar-how-should-united-states-protect-data/})
\end{itemize}

% %%%%%%%%%%%%%%%%%%%%%%%%%%%%%%%%%%%%%%%%%%
% %% optional
% \supplementary{The following are available online at www.mdpi.com/link, Figure S1: title, Table S1: title, Video S1: title.}
%
% % Only for the journal Methods and Protocols:
% % If you wish to submit a video article, please do so with any other supplementary material.
% % \supplementary{The following are available at www.mdpi.com/link: Figure S1: title, Table S1: title, Video S1: title. A supporting video article is available at doi: link.}

\vspace{6pt}

%%%%%%%%%%%%%%%%%%%%%%%%%%%%%%%%%%%%%%%%%%
\acknowledgments{All sources of funding of the study should be
disclosed. Please clearly indicate grants that you have received in
support of your research work. Clearly state if you received funds for
covering the costs to publish in open access.}

%%%%%%%%%%%%%%%%%%%%%%%%%%%%%%%%%%%%%%%%%%

%%%%%%%%%%%%%%%%%%%%%%%%%%%%%%%%%%%%%%%%%%

%%%%%%%%%%%%%%%%%%%%%%%%%%%%%%%%%%%%%%%%%%
%% optional

\input{"appendix.tex"}

%%%%%%%%%%%%%%%%%%%%%%%%%%%%%%%%%%%%%%%%%%
% Citations and References in Supplementary files are permitted provided that they also appear in the reference list here.

%=====================================
% References, variant A: internal bibliography
%=====================================
%\reftitle{References}
%\begin{thebibliography}{999}
% Reference 1
%\bibitem[Author1(year)]{ref-journal}
%Author1, T. The title of the cited article. {\em Journal Abbreviation} {\bf 2008}, {\em 10}, 142--149.
% Reference 2
%\bibitem[Author2(year)]{ref-book}
%Author2, L. The title of the cited contribution. In {\em The Book Title}; Editor1, F., Editor2, A., Eds.; Publishing House: City, Country, 2007; pp. 32--58.
%\end{thebibliography}

% The following MDPI journals use author-date citation: Arts, Econometrics, Economies, Genealogy, Humanities, IJFS, JRFM, Laws, Religions, Risks, Social Sciences. For those journals, please follow the formatting guidelines on http://www.mdpi.com/authors/references
% To cite two works by the same author: \citeauthor{ref-journal-1a} (\citeyear{ref-journal-1a}, \citeyear{ref-journal-1b}). This produces: Whittaker (1967, 1975)
% To cite two works by the same author with specific pages: \citeauthor{ref-journal-3a} (\citeyear{ref-journal-3a}, p. 328; \citeyear{ref-journal-3b}, p.475). This produces: Wong (1999, p. 328; 2000, p. 475)

%=====================================
% References, variant B: external bibliography
%=====================================
\reftitle{References}
\externalbibliography{yes}
\bibliography{mybibfile.bib}

%%%%%%%%%%%%%%%%%%%%%%%%%%%%%%%%%%%%%%%%%%
%% optional

%% for journal Sci
%\reviewreports{\\
%Reviewer 1 comments and authors’ response\\
%Reviewer 2 comments and authors’ response\\
%Reviewer 3 comments and authors’ response
%}

%%%%%%%%%%%%%%%%%%%%%%%%%%%%%%%%%%%%%%%%%%
\end{document}
